%% MUST PATCH footmisc.sty - in the definition of  \long\def\makefootnoteparagraph replace \hsize\columnwidth with \hsize\textwidth

\documentclass[10pt]{article}
\frenchspacing
\usepackage{fontspec}
\usepackage{polyglossia}
\setdefaultlanguage{magyar}
\setmainfont{Linux Libertine O}
\usepackage{lettrine}
\usepackage{multicol}
\usepackage[para,flushmargin]{footmisc}
\usepackage[b5paper, margin=1.5cm]{geometry}
\setlength{\columnseprule}{0.2pt}
\begin{document}
\section*{EVANGÉLIUM LUKÁCS SZERINT}
\begin{multicols}{2}
 \lettrine{1}{} \textsuperscript{1}Mivel már sokan megkísérelték rendben elbeszélni a köztünk végbement eseményeket,
 \let\thefootnote\relax\footnote{※ 1 • 1-4 Lukács a görög történetírókhoz hasonlóan ajánlással kezdi elbeszélését. Ebben említést tesz arról is, hogy milyen alaposan utánajárt az elbeszélt dolgoknak.
  -- Teofil személye ismeretlen számunkra. Mivel a jév jelentése >>istenszerező<<, vonatkozhat minden keresztényre.}
 \subsection*{ELŐSZÓ: 1,1-4}
\textsuperscript{2}amint előadták azt nekünk azok, akik kezdet óta szemtanúi és szolgái voltak az igének,
\textsuperscript{3}jónak láttam én is, miután mindennek elejétől fogva gondosan a végére jártam, neked, kegyelmes Teofil, sorrendben leírni,
\textsuperscript{4}hogy jól megismerd azon dolgoknak bizonyosságát, amelyekre téged oktattak.

 \subsubsection*{AZ ELŐTÖRTÉNET: 1,5-2,52}
 \paragraph*{Keresztelő János születésének hírüladása}

\textsuperscript{5}Volt Heródesnek, Júdea királyának napjaiban egy Zakariás nevű pap, Ábia papi osztályából.
 A felesége Áron leányai közül volt, és Erzsébetnek hívták.
\textsuperscript{6}Mindketten igazak voltak Isten előtt, feddhetetlenül éltek az Úrnak minden parancsa és rendelése szerint.
\textsuperscript{7}De nem volt gyermekük, minthogy Erzsébet magtalan volt, és már mindketten előhaladott korban voltak.
\textsuperscript{8}Történt pedig, hogy amikor osztályának rendjében papi szolgálatot teljesített az Isten előtt,
 \let\thefootnote\relax\footnote{• 8-10 a jeruzsálemi templomban naponta sorshúzással döntötték el, hogy ki végezze az egyes papi teendőket. Ezen a napon Zakariás végezheti a legfontosabb szertartást, a tömjénáldozatot az illatáldozat oltárán.
A nép eközben kint várakozott az áldásra, amelyet a szertartás befejeztével a pap adott (Szám 6,24-26)}
\textsuperscript{9}s a papi szolgálat rendje szerint sorsvetés útján rá került a tömjénezés sora, bement az Úr templomába,
\textsuperscript{10}miközben az egész népsokaság kívül imádkozott az illatáldozat órájában.
\textsuperscript{11}Akkor megjelent neki az Úr angyala, és megállt a tömjénoltár jobb oldalán.
\textsuperscript{12}Ennek láttára Zakariás zavarba jött, és félelem szállta meg.
\textsuperscript{13}Az angyal ezt mondta neki: »Ne félj, Zakariás, mert meghallgatást nyert könyörgésed; feleséged, Erzsébet fiút szül neked, és a nevét Jánosnak fogod hívni. { \footnotesize\textbf{\{Ter~17,19\}}}
\textsuperscript{14}Örömöd és vigasságod lesz ő, és sokan örülnek majd születésén.
\textsuperscript{15}Mert nagy lesz az Úr előtt; bort és részegítő italt nem iszik, és már anyja méhétől fogva betelik Szentlélekkel.
\textsuperscript{16}Izrael fiai közül sokakat fog Urukhoz, Istenükhöz téríteni.
\textsuperscript{17}Illés szellemével és erejével fog előtte járni, hogy az atyák szívét a fiakhoz fordítsa, a hitetleneket pedig az igazak okosságára, s így alkalmas népet készítsen az Úrnak.«
\textsuperscript{18}Ekkor Zakariás megkérdezte: »Hogyan győződjem meg erről? Hiszen én öreg vagyok, és a feleségem is előrehaladott már napjaiban.«
\textsuperscript{19}Az angyal ezt felelte neki: »Én Gábriel vagyok, aki az Isten színe előtt állok, és azért küldtek, hogy szóljak hozzád, és ezt az örömhírt meghozzam neked.
\textsuperscript{20}De íme, megnémulsz, és nem tudsz beszélni addig a napig, amikor ezek megtörténnek, mivel nem hittél szavaimnak, amelyek a maguk idejében beteljesednek.«
\textsuperscript{21}Ezalatt a nép várta Zakariást, és csodálkozott, hogy késik a templomban.
\textsuperscript{22}Amikor pedig kijött, nem tudott szólni hozzájuk, és megértették, hogy látomást látott a templomban; ő pedig intett nekik és néma maradt.

\textsuperscript{23}Amikor azután elteltek szolgálatának napjai, visszament a házába.  
\textsuperscript{24}E napok után Erzsébet, a felesége méhében fogant, s elrejtőzött öt hónapig, mondván:  
\textsuperscript{25}»Így cselekedett velem az Úr ezekben a napokban, amikor rámtekintett, hogy elvegye gyalázatomat az emberek előtt.«

\paragraph*{Jézus születésének hírüladása}
\textsuperscript{26}Isten pedig a hatodik hónapban elküldte Gábriel angyalt Galilea városába, amelynek Názáret a neve,  
\textsuperscript{27}egy szűzhöz, aki el volt jegyezve egy férfival. A neve József volt, Dávid házából, a szűz neve meg Mária.  
\textsuperscript{28}Bement hozzá az angyal, és így szólt: »Üdvözlégy, kegyelemmel teljes, az Úr van teveled.«  
\textsuperscript{29}Őt zavarba ejtette ez a beszéd, és elgondolkodott, hogy miféle köszöntés ez.  
\textsuperscript{30}Az angyal pedig folytatta: »Ne félj, Mária! Kegyelmet találtál Istennél.  
\textsuperscript{31}Íme, méhedben fogansz és fiút szülsz, és Jézusnak fogod nevezni.  
\textsuperscript{32}Nagy lesz ő, a Magasságbeli Fiának fogják hívni; az Úr Isten neki adja atyjának, Dávidnak trónját,  
\textsuperscript{33}és uralkodni fog Jákob házában mindörökké, és királyságának nem lesz vége«.  
\textsuperscript{34}Mária erre így szólt az angyalhoz: »Miképpen lesz ez, hiszen férfit nem ismerek?«  
\textsuperscript{35}Az angyal ezt felelte neki: »A Szentlélek száll rád, és a Magasságbeli ereje megárnyékoz téged; s ezért a Szentet, aki tőled születik, Isten Fiának fogják hívni.  
\textsuperscript{36}Íme, Erzsébet, a te rokonod is fiat fogant öregségében, és már a hatodik hónapban van, ő, akit magtalannak hívtak,  
\textsuperscript{37}mert Istennek semmi sem lehetetlen«.  
\textsuperscript{38}Mária erre így szólt: »Íme, az Úr szolgálóleánya, legyen nekem a te igéd szerint.« És eltávozott tőle az angyal.

 \paragraph*{Mária látogatása Erzsébetnél}
\textsuperscript{39}Mária pedig útra kelt azokban a napokban, és sietve elment a hegyek közé, Júda városába.  
\textsuperscript{40}Bement Zakariás házába, és köszöntötte Erzsébetet.  
\textsuperscript{41}És történt, hogy amint Erzsébet meghallotta Mária köszöntését, felujjongott méhében a magzat, és Erzsébet eltelt Szentlélekkel.  
\textsuperscript{42}Hangosan felkiáltott: »Áldott vagy te az asszonyok között, és áldott a te méhednek gyümölcse!  
\textsuperscript{43}De hogyan történhet velem az, hogy az én Uramnak anyja jön hozzám?  
\textsuperscript{44}Mert íme, amint fülemben felhangzott köszöntésed szava, felujjongott a magzat méhemben.  
\textsuperscript{45}És boldog, aki hitt, mert be fog teljesedni, amit az Úr mondott neki.«

\newlength{\saveleftmargini} % define a temp variable for the original margin
\setlength{\saveleftmargini}{\leftmargini} % write the original margin in this variable
\setlength{\leftmargini}{2em} % set the left margin to zero
\textsuperscript{46}Mária erre így szólt:
\disablehyphenation
\begin{flushleft}
\begin{verse}
»Magasztalja lelkem az Urat, \\
\textsuperscript{47}és szívem ujjong megváltó Istenemben, \\
\textsuperscript{48}mert tekintetre méltatta szolgálója alázatosságát. \\
Íme, mostantól fogva boldognak hirdet engem minden nemzedék, \\
\textsuperscript{49}mert nagy dolgot cselekedett velem a Hatalmas\\ és Szent az ő Neve. \\
\textsuperscript{50}Irgalma nemzedékről nemzedékre \\ azokra száll, akik őt félik. \\
\textsuperscript{51}Hatalmas dolgokat művelt karja erejével, \\ szétszórta a gondolataikban kevélykedőket. \\
\textsuperscript{52}Hatalmasokat levetett a trónról, \\ és kicsinyeket felemelt. \\
\textsuperscript{53}Éhezőket betöltött jókkal, \\ és üresen bocsátott el gazdagokat. \\
\textsuperscript{54}Felkarolta szolgáját, Izraelt, \\ megemlékezve irgalmasságáról, \\
\textsuperscript{55}amint megmondta atyáinknak, \\ Ábrahámnak és utódainak mindörökre«.
\end{verse}
\end{flushleft}
\enablehyphenation
\setlength{\leftmargini}{\saveleftmargini}
\textsuperscript{56}És Mária nála maradt mintegy három hónapig, azután visszatért házába.
\paragraph*{Keresztelő János születése}
\textsuperscript{57}Azután eljött az ideje, hogy Erzsébet szüljön; és fiút szült.  
\textsuperscript{58}Meghallották szomszédai és rokonai, hogy az Úr nagy irgalmasságot cselekedett vele, és együtt örvendeztek vele.  
\textsuperscript{59}Történt pedig, hogy a nyolcadik napon eljöttek körülmetélni a gyermeket, és apja nevéről Zakariásnak akarták nevezni.  
\textsuperscript{60}De az anyja így szólt: »Semmiképpen sem, hanem Jánosnak fogják hívni.«  
\textsuperscript{61}Erre azt mondták: »De hiszen senki sincs a rokonságodban, akit így neveznének.«  
\textsuperscript{62}Megkérdezték tehát az apját, hogyan akarja őt nevezni.  
\textsuperscript{63}Ő pedig írótáblát kért, és ezeket a szavakat írta rá: »János a neve.« Mindnyájan elcsodálkoztak.  
\textsuperscript{64}Erre azonnal megnyílt a szája és a nyelve, megszólalt, és magasztalta Istent.  
\textsuperscript{65}Félelem szállta meg összes szomszédjukat, s e dolgok híre elterjedt Júdea egész hegyvidékén.  
\textsuperscript{66}Mindannyian, akik hallották, a szívükbe vésték ezt, és kérdezték: »Mi lesz ebből a gyermekből?« Mert az Úr keze volt vele. \\  
\textsuperscript{67}Apja pedig, Zakariás, betelt Szentlélekkel és így jövendölt: \\  
\textsuperscript{68}»Áldott az Úr, Izrael Istene ,, mert meglátogatta és megváltotta az ő népét. \\  
\textsuperscript{69}Az üdvösség jelét támasztotta nekünk, \\ Dávidnak, az ő szolgájának házában,  
\textsuperscript{70}amint megmondta szentjeinek ajkával, \\ ősidőktől fogva prófétái által. \\  
\textsuperscript{71}Megmentett minket ellenségeinktől, \\ és mindazok kezéből, akik gyűlölnek minket;  
\textsuperscript{72}hogy irgalmasságot cselekedjék atyáinkkal, \\ és megemlékezzék szent szövetségéről, \\  
\textsuperscript{73}az esküről, melyet Ábrahám atyánknak esküdött, \\ hogy majd megadja nekünk,  
\textsuperscript{74}hogy ellenségeink kezéből megszabadulva, \\ félelem nélkül szolgáljunk neki, \\  
\textsuperscript{75}szentségben és igazságban színe előtt \\ életünknek minden napján. \\  
\textsuperscript{76}Téged pedig, gyermek, \\ a Magasságbeli prófétájának fognak hívni: \\ mert az Úr színe előtt fogsz járni, \\ hogy előkészítsd az ő útját,  
\textsuperscript{77}és népét az üdvösség ismeretére tanítsd, \\ bűneik bocsánatára, \\  
\textsuperscript{78}Istenünk mélységes irgalmából, \\ amellyel meglátogatott minket a magasságban felkelő, \\  
\textsuperscript{79}hogy fényt hozzon azoknak, \\ akik sötétségben és a halál árnyékában ülnek,, s hogy lépteinket a békesség útjára igazítsa«.  
\textsuperscript{80}A gyermek pedig növekedett és erősödött lélekben, és a pusztában volt addig a napig, amíg nyilvánosan fel nem lépett Izraelben.

\lettrine{2}{}
\textbf{Jézus születése} \textsuperscript{1}Történt pedig azokban a napokban: Rendelet ment ki Augusztusz császártól, hogy írassék össze az egész földkerekség.
\textsuperscript{2}Ez az első összeírás akkor történt, amikor Szíriát Kvirínusz kormányozta.  
\textsuperscript{3}El is ment mindenki, hogy összeírják, mindenki a maga városába.  
\textsuperscript{4}Fölment tehát József is Galileából, Názáret városából Júdeába, Dávid városába, amelyet Betlehemnek hívnak, mert Dávid házából és nemzetségéből való volt,  
\textsuperscript{5}hogy összeírják Máriával, eljegyzett feleségével, aki áldott állapotban volt.  
\textsuperscript{6}Amikor ott voltak, eljött az ideje, hogy szüljön,  
\textsuperscript{7}és megszülte elsőszülött fiát. Pólyába takarta és jászolba fektette, mert nem kaptak helyet a szálláson. \\  
\textsuperscript{8}Azon a vidéken pásztorok tanyáztak, és őrizték nyájukat az éjszakában.  
\textsuperscript{9}Egyszer csak ott termett mellettük az Úr angyala, és az Úr fényessége körülragyogta őket. Nagy félelem vett erőt rajtuk.  
\textsuperscript{10}Az angyal ezt mondta nekik: »Ne féljetek! Íme, nagy örömet hirdetek nektek, melyben része lesz az egész népnek.  
\textsuperscript{11}Ma született nektek az Üdvözítő, az Úr Krisztus, Dávid városában.  
\textsuperscript{12}Ez lesz a jel számotokra: találni fogtok egy kisdedet pólyába takarva és jászolba fektetve.«  
\textsuperscript{13}Ekkor azonnal mennyei sereg sokasága vette körül az angyalt, és dicsérte Istent: \\  
\textsuperscript{14}»Dicsőség a magasságban Istennek, \\ és a földön békesség a jóakaratú emberekben!« \\  
\textsuperscript{15}És történt, hogy amikor az angyalok visszatértek tőlük a mennybe, a pásztorok így szóltak egymáshoz: »Menjünk át Betlehembe, és lássuk azt a dolgot, ami történt, s amelyet az Úr hírül adott nekünk.«  
\textsuperscript{16}Elmentek tehát sietve, és megtalálták Máriát és Józsefet, és a jászolban fekvő kisdedet.  
\textsuperscript{17}Amikor meglátták őket, elhíresztelték azt, amit a gyermek felől hallottak.  
\textsuperscript{18}És mindnyájan, akik hallották, csodálkoztak azon, amiről a pásztorok beszéltek nekik.  
\textsuperscript{19}Mária pedig megjegyezte mindezeket a dolgokat, és el-elgondolkodott rajtuk szívében. \\  
\textsuperscript{20}A pásztorok pedig visszatértek, magasztalták és dicsérték Istent mindazokért a dolgokért, amiket hallottak és láttak úgy, ahogy megmondták nekik.
\paragraph*{Simeon és Anna tanúsága Jézusról}
\textsuperscript{21}Amikor elérkezett a nyolcadik nap, hogy körülmetéljék a gyermeket, a Jézus nevet adták neki, úgy, amint az angyal nevezte, mielőtt anyja méhében fogantatott. \\  
\textsuperscript{22}Mikor pedig elteltek tisztulásuk napjai, Mózes törvénye szerint felvitték őt Jeruzsálembe, hogy bemutassák az Úrnak,  
\textsuperscript{23}amint az Úr törvényében írva van: »Minden elsőszülött fiúgyermek az Úrnak legyen szentelve« ,  
\textsuperscript{24}és hogy áldozatot mutassanak be, amint az Úr törvénye mondja: »Egy pár gerlicét vagy két galambfiókát«.  
\textsuperscript{25}Élt pedig Jeruzsálemben egy ember, Simeon volt a neve, igaz és istenfélő férfiú, aki várta Izrael vigasztalását, és a Szentlélek volt rajta.  
\textsuperscript{26}A Szentlélek kijelentette neki, hogy halált nem lát, amíg meg nem látja az Úr Felkentjét.  
\textsuperscript{27}Ekkor a Lélek ösztönzésére a templomba ment. Amikor szülei bevitték a gyermek Jézust, hogy a törvény szokása szerint cselekedjenek vele,  
\textsuperscript{28}karjaiba vette őt, és Istent magasztalva így szólt: \\  
\textsuperscript{29}»Most bocsátod el, Uram, szolgádat \\ a te igéd szerint békességben, \\  
\textsuperscript{30}mert látták szemeim \\ üdvösségedet,  
\textsuperscript{31}melyet minden nép \\ színe előtt készítettél,  
\textsuperscript{32}világosságul a nemzetek megvilágosítására és dicsőségére népednek, Izraelnek«.  
\textsuperscript{33}Apja és anyja csodálkoztak mindazon, amit róla mondtak.  
\textsuperscript{34}Simeon megáldotta őket, anyjának, Máriának pedig ezt mondta: »Íme, sokak romlására és feltámadására lesz ő Izraelben; jel lesz, melynek ellene mondanak  
\textsuperscript{35}– és a te lelkedet tőr járja át –, hogy nyilvánosságra jussanak sok szív gondolatai.« \\  
\textsuperscript{36}Volt egy Anna nevű prófétaasszony is, Fánuel leánya, Áser törzséből. Nagyon előre haladt már napjaiban, miután férjével hét esztendeig élt szüzessége után;  
\textsuperscript{37}nyolcvannégy éves özvegy volt, és nem vált meg a templomtól, böjtöléssel és imádsággal szolgált ott éjjel és nappal.  
\textsuperscript{38}Ő is odajött ugyanabban az órában, dicsérte Istent, és beszélt róla mindazoknak, akik várták Jeruzsálem megváltását. \\  
\textsuperscript{39}Miután mindent elvégeztek az Úr törvénye szerint, visszatértek Galileába, a városukba, Názáretbe.  
\textsuperscript{40}A gyermek pedig növekedett és erősödött, telve bölcsességgel, és az Isten kegyelme volt rajta.
\paragraph*{A tizenkét éves Jézus a templomban}
\textsuperscript{41}A szülei pedig minden évben elmentek Jeruzsálembe a Húsvét ünnepére.  
\textsuperscript{42}Mikor azután tizenkét esztendős lett, fölmentek mindnyájan Jeruzsálembe az ünnepi szokás szerint.  
\textsuperscript{43}Amikor elteltek az ünnepnapok és már visszatérőben voltak, a gyermek Jézus ottmaradt Jeruzsálemben, és nem vették észre a szülei.  
\textsuperscript{44}Úgy gondolták, hogy az úti társaságban van. Megtettek egy napi utat, s akkor keresték őt a rokonok és az ismerősök között.  
\textsuperscript{45}Mivel nem találták, visszatértek Jeruzsálembe, hogy megkeressék.  
\textsuperscript{46}És történt, hogy három nap múlva megtalálták őt a templomban, amint a tanítók közt ült, hallgatta és kérdezte őket.  
\textsuperscript{47}Mindnyájan, akik hallották, csodálkoztak okosságán és feleletein.  
\textsuperscript{48}Mikor meglátták őt, elcsodálkoztak, és anyja ezt mondta neki: »Fiam! Miért tetted ezt velünk? Íme, apád és én bánkódva kerestünk téged.«  
\textsuperscript{49}Ő pedig ezt felelte nekik: »Miért kerestetek engem? Nem tudtátok, hogy nekem az én Atyám dolgaiban kell lennem?«  
\textsuperscript{50}De ők nem értették meg, amit nekik mondott. \\  
\textsuperscript{51}Akkor hazatért velük. Elment Názáretbe, és engedelmeskedett nekik. Anyja megőrizte szívében mindezeket a szavakat.  
\textsuperscript{52}Jézus pedig növekedett bölcsességben, korban és kedvességben Isten és az emberek előtt. \\ 
\end{multicols}

\end{document}